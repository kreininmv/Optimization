 \section{Convex sets}

\subsection{Problem №1}

Show that the convex hull of the $\mathbf{S}$ set is the intersection of all convex sets containing $\mathbf{S}$

\underline{\textbf{Solution:}}

Firstly, we will prove that if A is convex set, then any convex combination $x_1, .., x_n \in A$ will belong to A.

For $n = 1$ it's trivial.

Assume that this is true for any convex combination of n-1 points. Point $x = \sum\limits_{j=1}^n\alpha_j x_j$, $\sum\limits_{j=1}^n\alpha_j = 1$, $\alpha_1, ..., \alpha_n \in \mathds{R}$ and $n > 1$. Between $\alpha_1, ..., \alpha_n$ we can find $\alpha$ that will not be equal 1. Without detracting from the community we consider that is $\alpha \not = 1$.
\begin{equation*}
    \overline{\alpha_j} = \frac{\alpha_j}{1-\alpha_1}\text{, }j = 2, ..., n
\end{equation*}
Because $    \sum\limits_{j=2}^n\overline{\alpha_j} = 1 $
and from induction we can get that $ \overline{x} = \sum\limits_{j=1}^n\overline{\alpha_j}x_j \in A $
Then from convex of A we get:
\begin{equation*}
    x = \sum\limits_{j=1}^n\alpha_jx_j = \alpha_1x_1 + (1-\alpha_1)\sum\limits_{j=1}^n\frac{\alpha_j}{1-\alpha_1}x_j = \alpha_1 x_1 + (1-\alpha_1)\overline{x} \in A
\end{equation*}

From the proven it follows that if some convex set contains A, then it contains any a convex combination of points from A, which means it contains convex hull of A. Let's show that convex hull of A is convex set, in that case it coincides with the intersection of all convex sets containing A. Let's take random points from convex hull A. 
\begin{equation*}
    x = \sum\limits_{j=1}^n \alpha_jx_j \text{, } y = \sum\limits_{j = 1}^m \beta_jy_j
\end{equation*}
For any $\alpha \in [0, 1]$, we get:
\begin{equation*}
    (1-\alpha)x + \alpha y =\sum\limits_{j=1}^n(1-\alpha)\alpha_jx_j + \sum\limits_{j=1}^m\alpha \beta_j y_j
\end{equation*}

\begin{equation*}
    \sum\limits_{j=1}^n(1-\alpha)\alpha_j + \sum\limits_{j=1}^m\alpha \beta_j = (1-\alpha) + \alpha = 1
\end{equation*}
We get convex combination of points $x_1, ..., x_n, y_1, ..., y_m$, which belongs to convex hull of A. 

WOHOO!!!


%\subsection{Problem №2}
%Show that the set of directions of the strict local descending of the differentiable function in a point is a convex cone.

%\underline{\textbf{Solution:}}

\subsection{Problem №3 }
Prove, that if S is convex, then $S + S = 2S$. Give an counterexample in case, when S -- is not convex.

\underline{\textbf{Solution:}}
\newline
$\forall \alpha \in [0, 1]$ $\forall (x, y) \in 2S \hookrightarrow \alpha \cdot (x, y) + (1 - \alpha) \cdot (x, y) \in S$
\newline
Let's rewrite this expression: $(\alpha \cdot x, \alpha \cdot y) + ( (1 - \alpha) \cdot x, (1 - \alpha) \cdot y) \in 2S$
And it's right because $x \in S$ and S is convex set, the same for y. Due to that $2S = S + S$, and $\alpha \cdot x + (1-\alpha)\cdot x \in S$, $\alpha \cdot y + (1-\alpha)\cdot y \in S$ follows that $2S$ - convex set.

\subsection{Problem №4}
Let $x \in \mathds{R}$ is a random variable with a given probability distribution of $\mathds{P}(x = a_i) = p_i$, where $i = 1, ..., n,$ and $a_1 < ... < a_n$. It is said the probability vector of outcomes of $p \in \mathds{R}^n$ belongs to the probabilistic simpex, i.e. $P = \{ p | \textbf{1}^Tp = 1, p \succcurlyeq 0\} = \{ p | p_1 + ... + p_n = 1, p_i \geq 0 \}$.
\newline
Determine if the following sets of p are convex:
\begin{itemize}
    \item $\alpha < \mathds{E} f(x) < \beta$, where $\mathds{E}f(x)$ stands for expected value of $f(x) : \mathds{R} \xrightarrow{} \mathds{R}$, i.e. $\mathds{E}f(x) = \sum\limits_{i=1}^n p_i\cdot f(a_i)$

    \item $\mathds{E}x^2 \leq \alpha$
    \item $\mathds{V}x \leq \alpha$
\end{itemize}

\underline{\textbf{Solution:}}

\underline{\textbf{A. }} It's right because we reduce constraints on p as constraints on the half-space, it will follow from this that the set is convex.

\begin{equation*}
    \alpha < \mathds{E} f(x) = \sum\limits_{i=1}^n p_if(a_i) < \beta
\end{equation*}
From the geometry it is half-space and convex, and it means that our set is also convex.


\underline{\textbf{B. }}
Here, we have the same idea like in \underline{\textbf{A.}}. We reduce constraints on p as constraints on the half-space, it will follow this that set is convex.
\begin{equation*}
\mathds{E} x^2 = \sum\limits_{i = 1}^n p_i a_i^2 \leq \alpha
\end{equation*}

\underline{\textbf{C.}}
\begin{equation*}
    0 \leq \mathds{V}x = \mathds{E}x^2 - (\mathds{E} x)^2 = \sum\limits_{i = 1}^n p_i a_i^2 - \left( \sum\limits_{i = 1}^n p_ia_i \right)^2 = -p^TXp + d^Tp \leq \alpha
\end{equation*}
where $d_i = a_i$ and $X = aa^T$, $X \succ 0$. This is a parabola with branches down in multidimensional space. Under the graph of a parabola is a convex set. And we also get that a convex set cut off by a hyperplane is a convex set.

\underline{\textbf{Answer:}}
\textbf{A.-C.} convex

\subsection{Problem №5}
Let $S \subset \mathds{R}^n$ is a set of solutions to the quadratic inequality:

\begin{equation*}
    S = \{x \in \mathds{R}^n \text{ | } x^TAx + b^Tx + c \leq 0 \}\text{; } A \in \mathds{S}^n, b \in \mathds{R}^n, c \in \mathds{R}
\end{equation*}
\begin{itemize}
    \item Show that if $A \succcurlyeq 0$, S is convex. Is the opposite true?
    \item Show that intersection of S with the hyperplane defined by the $g^Tx + h = 0$, $g \not = 0$ is convex if $A + \lambda gg^T \succcurlyeq 0$ for some $\lambda \in \mathds{R}$. Is the opposoite true?
\end{itemize}

\underline{\textbf{Solution:}}

\underline{\textbf{A.)}}
$x, y \in S$, $\theta \in [0, 1]$. If $\theta = 0$ or 1, it's trivial.

\begin{equation*}
    \theta x^TAx + \theta b^Tx + \theta c \leq 0
\end{equation*}

\begin{equation*}
    (1-\theta) y^TAy + (1-\theta) b^Ty + (1-\theta) c \leq 0
\end{equation*}

\begin{equation*}
    (\theta x + (1-\theta)y)^TA(\theta x + (1-\theta)y) + b^T(\theta x + (1-\theta)y) + c \leq 0
\end{equation*}

\begin{equation*}
    \theta^2 x^TAx + \theta(1-\theta)\left(y^TAx + x^TAy \right) + (1-\theta)^2 y^TAy + \theta b^Tx + (1-\theta)b^Ty + c \leq 0
\end{equation*}

Okay, lets' add $\theta x^TAx + (1-\theta)y^TAy$ and subtract it. 
\newline
$z = \theta^2 x^TAx + \theta(1-\theta)\left(y^TAx + x^TAy \right) + (1-\theta)^2 y^TAy - \theta x^TAx - (1-\theta)y^TAy $

\begin{equation*}
    z + \left(\theta x^TAx + \theta b^Tx + \theta c \right) + \left((1-\theta)y^TAy + (1-\theta)b^Ty + (1-\theta)c \right) \leq 0
\end{equation*}
From that
\begin{equation*}
    z = \theta^2 x^TAx + \theta(1-\theta)\left(y^TAx + x^TAy \right) + (1-\theta)^2 y^TAy - \theta x^TAx - (1-\theta)y^TAy  \leq 0
\end{equation*}

\begin{equation*}
    \theta(1-\theta) x^TAx + (1-\theta)\theta y^TAy \geq \theta(1-\theta)(y^Tax + x^TAy)
\end{equation*}

\begin{equation*}
    x^TA(x-y) + (y-x)^TAy \geq 0
\end{equation*}
It's right because $A \succcurlyeq$. 

\underline{\textbf{B.)}}

\underline{\textbf{Answer:}}








