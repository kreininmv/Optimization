\section{Convex functions}

\subsection{Problem №1}
Is $f(x) = -x \cdot lnx - (1-x)ln(1-x)$ convex?

\underline{\textbf{Solution:}}

\begin{equation*}
    \nabla^2 f(x) = \frac{\partial}{\partial x} \left( -lnx - 1 + ln(1-x) + 1\right) = \frac{-1}{x} - \frac{-1}{1-x} = \frac{-1}{x(1-x)} < 0
\end{equation*}
because $x \in (0, 1)$ it is right. From this we get that f(x) is concave function, but not convex.

\underline{\textbf{Answer:}} No, it's not convex function, it's concave function.

\subsection{Problem №2}
Let x be a real variable with the values $a_1 < a_2 < ... < a_n$ with probabilities $\mathds{P} (x = a_i) = p_i$. Derive the convexity or concavity of the following functions from p on the set of $\left\{p \text{ | } \sum\limits_{i = 1}^n p_i = 1, \text{ } p_i \geq 0 \right)$

\underline{\textbf{Solution:}}
We know that linear function: $a^Tx + b$ convex and concave function at the same time.

\textbf{1.} $\mathds{E}x = \sum\limits_{i = 1}^n a_i \cdot p_i$ -- that's linear function, therefore math expectation is convex and concave function.

\textbf{2.} $\mathds{P} \{x \geq \alpha \} = \sum\limits_{i: a_i \geq \alpha} p_i$ -- it's linear function, therefore it's convex and concave function.

\textbf{3.} $\mathds{P} \{\alpha \leq x \leq \beta \} = \sum\limits_{i: \alpha \leq a_i \leq b_i}^n p_i$, we get that is convex and concave function.

\textbf{4.} $\sum\limits_{i = 1}^n p_i \log p_i$.

Okay, let's check is $f(x) = x \log x $ - convex

$\nabla^2f(x) = (x \log x)^{''} = (\log x + 1)^{'} = \frac{1}{x} > 0$, because $x > 0$ - yep, it's convex function. And our function is non-negative sum of convex function, then our function is convex function.

\textbf{5.} $\mathds{V} x = \mathds{E} (x-\mathds{E}x)^2 = \mathds{E}x^2 - \left( \mathds{E} x\right)^2$

Okay, let's take counterexample for this function:
\begin{enumerate}
    \item  $p_a = (1, 0)$, $x = (0, 1)$, $\mathds{V}x = 1 \cdot 1^2 - (1 \cdot 1)^2 = 0$

    \item $p_b = (0, 1)$, $x = (0, 1)$, $\mathds{V}x = 0 \cdot 1^2 + 1 \cdot 0^2 - (1 \cdot 0 + 0 \cdot 1)^2 = 0$

    \item $p_c = 0.5\cdot p_a + 0.5 \cdot p_b = (\frac{1}{2}, \frac{1}{2})$, $x = (0, 1)$,
    $\mathds{V}x = 0.5 \cdot 1^2 - (0.5 \cdot 1)^2 = 0.25$
\end{enumerate}

By definition of convex function the following equality is right:
\newline
\begin{equation*}
f(\theta x + (1-\theta)y ) \leq \theta f(x) + (1-\theta) f(y)    
\end{equation*}
But:
\begin{equation*}
    f(0.5p_a + 0.5p_b)=\frac{1}{4} \leq \frac{1}{2}f(p_a) + \frac{1}{2}f(p_b) = 0 + 0 = 0
\end{equation*}
And we get that it's not convex function, it's concave function.

\textbf{6.} \textbf{quatile}(x) = $\inf \{ \beta$ | $\mathds{P} \{x \leq \beta \} \geq 0.25\}$ 

$\textbf{quartile}$ is not continuous function, because x can take discrete values, then it is defined on a discrete set of points, that is not convex. It means that function is not convex and concave.

\underline{\textbf{Answer:}}
\textbf{a-c} convex and concave function,  \textbf{d.} convex function, \textbf{e.} 
concave function. \textbf{f.} not concave, not convex function.

\subsection{Problem №3}
Show, that $f(A) = \lambda_{max}(A)$ -- is convex, if $A \in S^n$

\underline{\textbf{Solution:}}
Okay, let's show that's is false:

Let's take: 
\begin{equation*}
    A = \begin{bmatrix}
    -8 & 16 \\
    60 & 4 
\end{bmatrix} \text{, }     B = \begin{bmatrix}
    2 & 40 \\
    20 & -2
\end{bmatrix}
\end{equation*}

\begin{equation*}
    \lambda_{max}(0.5 A + 0.5B) = \lambda_{max} \left( \begin{bmatrix}
        -3 & 28 \\
        40 & 1
    \end{bmatrix} \right) \leq 0.5 \lambda_{max}\left(\begin{bmatrix}
    -8 & 16 \\
    60 & 4 
\end{bmatrix} \right) + 0.5 \lambda_{max}\left( \begin{bmatrix}
    2 & 40 \\
    20 & -2
\end{bmatrix}\right)
\end{equation*}

$ \lambda_{max} \left( \begin{bmatrix}
        -3 & 28 \\
        40 & 1
    \end{bmatrix} \right)$, $\lambda_{max}\left( \begin{bmatrix}
        -3 & 28 \\
        40 & 1
    \end{bmatrix} \right) = 2(\sqrt{249} - 1)$, $\lambda_{max} \left(\begin{bmatrix}
    -8 & 16 \\
    60 & 4 
\end{bmatrix} \right) = 2\sqrt{201}$

\begin{equation*}
    \sqrt{1124} - 1 \approx 32.52 \leq \sqrt{249} - 1 + \sqrt{201} \approx 28.96
\end{equation*}

We see that the inequality for a convex function does not hold. Hence, it is not a convex function
\underline{\textbf{Answer:}}

\subsection{Problem №4}
Prove that, $f(X) = - \log \det X$ is convex on $X \in \mathds{S}_{++}^n$

\underline{\textbf{Solution:}}
\begin{equation*}
df(x) = - \frac{1}{detX} detX \langle X^{-T}, dX \rangle = - \langle X^{-T}, dX \rangle
\end{equation*}

\begin{equation*}
    d^2f(x) = -d(tr(X^{-1}, dX_1) = -tr(d(X^{-1})dX_1) = tr(X^{-1}dX_2X^{-1}dX_2)
\end{equation*}
For a reason that $X^{-1}, dX_1, dX_2 \in \mathds{S}_{++}^n$ trace is positive, the hessian of f(x) is positive, then we get that f(x) is convex function on $\mathds{S}_{++}^n$


\subsection{Problem №5}

Prove, that adding $\lambda ||x||_2^2$ to any convex function $g(x)$ ensures strong convexity function of a resulting function $f(x) = g(x) + \lambda ||x||_2^2$. Find the constant of the strong convexity $\mu$.

\underline{\textbf{Solution:}}
\begin{equation*}
    \frac{\partial^2}{\partial x_l \partial x_k} \sum\limits_{i = 1}^n x_i = \frac{\partial}{\partial x_l} 2x_k = 0
\end{equation*}
\begin{equation*}
    \frac{\partial^2}{(\partial x_k)} \sum\limits_{i = 1}^n x_i = \frac{\partial}{\partial x_k} 2x_k = 2
\end{equation*}
The hessian of $||X||_2^2$ is $2 I$. Then $\nabla^2 f(x) = \nabla^2 g(x) + \lambda \cdot 2I \succcurlyeq \lambda I$. It's right because g(x) is convex function.

\subsection{Problem №6}
Study the following function of two variables $f(x, y) = e^{xy}$.
\begin{enumerate}
    \item[a.] Is this function convex?
    \item[b.] Prove, that this function will be convex on the line $x=y$.
    \item[c.] Find another set in $\mathds{R}^2, on which this function will be convex$
\end{enumerate}

\underline{\textbf{Solution:}}
\begin{enumerate}
    \item[a.] No this function is not convex on $\mathds{R}^2$, because the hessian equals $-(1+2xy)e^{2xy}$ and it is less than zero for $x = 0, y = 1$.
    \item[b.] $f(x, x) = e^{x^2}$, $\nabla^2 f(x) = 2(1+2x^2)e^{x^2} > 0$, $\forall x \in \mathds{R}$, and from that we get that $f(x)$ is strong convex function.
    \item[c.] Let's take $y = x^3$, $f(x, x) = e^{x^4}$, $\nabla^2 f(x) = (12x^2 + 16x^6)e^{x^4} \geq 0 $, $\forall x \in \mathds{R}$
\end{enumerate}

\underline{\textbf{Answer:}} 
\textbf{a.} No, this function is not convex on $\mathds{R}^2$, \textbf{b.} proved, \textbf{c.} $y = x^3$

%\subsection{Problem №7}
%Show, that the following function is convex on  the set of all positive denominators:
%\begin{equation*}
 %   f(x) \frac{1}{x_1 - \frac{1}{x_2 - \frac{1}{x_3 - \frac{1}{...}}}}, \text{ , } x \in \mathds{R}^n
%\end{equation*}