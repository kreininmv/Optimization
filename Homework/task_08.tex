\section{General optimization problem}
\subsection{Problem №1}
Consider the problem of projection some point $y \in \mathds{R}^n, y \not \in \Delta^n$ onto the unit simplex $\Delta^n$. Find 2 ways to solve the problem numerically and compare them in terms of the total computational time, memory requirements and iteration number for $n = 10, 100, 1000$
\begin{equation*}
    ||x-y||_2^2  \xrightarrow{} \min\limits_{x \in \mathds{R}^n}
\end{equation*}

\begin{equation*}
   \text{s.t. } \mathds{1}^Tx = 1, x \succeq 0
\end{equation*}

\underline{\textbf{Solution:}} there is no analytical solution. I had no time for do it. But I think that problem is $O(n^3)$.

\subsection{Problem №2}
Give an explicit solution of the following LP.
\begin{equation*}
    c^T x  \xrightarrow{} \min\limits_{x \in \mathds{R}^n}
\end{equation*}

\begin{equation*}
   \text{s.t. } Ax = b
\end{equation*}

\underline{\textbf{Solution:}}

Secondly, we can write: $L(x, \nu) = c^Tx + \nu(Ax-b)$, then write KKT conditions:
\begin{equation*}
    \begin{cases}
    \nabla_x L = c + A^T\lambda = 0 \\
    \nabla_{\nu} L = Ax - b =0   \\
    \nu \succeq 0 
    \end{cases}
\end{equation*}


It's convex optimization problem all functions are affine. 

\begin{enumerate}
    \item Firstly, if there is no solution to $Ax = b$, then budget set is empty and $p^* = +\infty$.
    \item Secondly, if there is solution to $Ax = b$, then $x^* = A^{\dag}b$ and $p^* = c^TA^{\dag}b$
\end{enumerate}

\underline{\textbf{Answer:}}
\begin{equation*}
    \begin{cases}
    c^TA^{\dag}b \text{, if there is no solution for } Ax = b \\
    + \infty \text{, otherwise} \\
    \end{cases}
\end{equation*}


\subsection{Problem №3}
Give an explicit solution of the following LP.

\begin{equation*}
    c^T x  \xrightarrow{} \min\limits_{x \in \mathds{R}^n}
\end{equation*}

\begin{equation*}
   \text{s.t. } \mathds{1}^Tx = 1, x \succeq 0
\end{equation*}
This problem can be considered as a simplest portfolio optimization problem.

\underline{\textbf{Solution:}}
Lagrangian:
\begin{equation*}
    L(x, \nu, \lambda) = c^Tx + \nu(\mathds{1}^Tx - 1) - \lambda^Tx
\end{equation*}
Let's write KKT conditions:
\begin{equation*}
    \begin{cases}
    \nabla_x L = c + \nu \mathds{1} - \lambda = 0 \\
    \nabla_{\nu} L = \mathds{1}^Tx - 1 =0   \\
    \lambda \succeq 0 \\
    \lambda_k x_k = 0, k \in \overline{1, n} \\
    x \succeq 0 
    
    \end{cases}
\end{equation*}


Okay, we can find minimal component of c ($c_k)$. It's clear that such component exists. And if we take $(\mathds{1}^Tx)c_k \leq c^Tx$ , for all x which satisfy for conditions. 

And we easily get: $p^* = c_k$. It's funny that we need to invest all our money in only one asset, but it's clearly true if we know and sure in all outcomes. Such $\lambda$ and $\nu$ exist that satisfy KKT, $\nu^* = - c_k$, $\lambda_i^* = c_i - c_k, i \in \overline{1, n}$

\underline{\textbf{Answer:}} $p^* = c_k$, $x^* = (0, ..., 1, ...0)$ 

\subsection{Problem №4}
Give an explicit solution of the following LP.

\begin{equation*}
    c^T x  \xrightarrow{} \min\limits_{x \in \mathds{R}^n}
\end{equation*}

\begin{equation*}
   \text{s.t. } \mathds{1}^Tx = \alpha, 0 \preccurlyeq x \preccurlyeq 1
\end{equation*}
where $\alpha$ is an integer between 0 and n. What happens if $\alpha$ is not an integer (but satisfies $0 \leq  \alpha \leq n)$? What if we change the equality to an inequality $\mathds{1}^Tx \leq \alpha$?

\underline{\textbf{Solution:}}
Lagrangian:
\begin{equation*}
    L(x, \nu, \lambda_1, \lambda_2) = c^Tx + \nu(\mathds{1}^Tx - \alpha) + \lambda_1^T(x - \mathds{1}) + \lambda_2^T(-x)
\end{equation*}
KKT conditions:
\begin{equation*}
    \begin{cases}
    \nabla_x L = c + \nu \mathds{1} + \lambda_1 - \lambda_2 = 0 \\
    \nabla_{\nu} L = \mathds{1}^Tx - 1 =0   \\
    \lambda_1 \succeq 0 \\
    \lambda_2 \succeq 0 \\
    \lambda_{1_k} (x_k - 1) = 0, k \in \overline{1, n} \\
    \lambda_{2_k} x_k = 0, k \in \overline{1, n} \\
    1 \succeq x \succeq 0
    \end{cases}
\end{equation*}
KKT conditions will be necessary and sufficient because Slater's condition is met.

If $\alpha$ is integer, then we consider without begging for generality without begging for generality (from previous task) $k = 0$, and the smallest components come first. 

We take first $\alpha$ components of c: $c_1, ..., c_{\alpha}$ and $x_i = 1$, $i \in \overline{1, \alpha}$, $x_k = 0$, $k \in \overline{\alpha + 1, n}$. $p^* = \sum\limits_{i=1}^\alpha c_i$. 

$\nu^* = -c_{\alpha}$; $\lambda_{1_i}^* = 0, i \in \overline{1, \alpha}$; $\lambda_{1_i} = -(c_i + c_{\alpha}), i \in \overline{\alpha + 1, n}$;
$\lambda_{2_i}^* = 0, i \in \overline{1, \alpha}$; $\lambda_{2_i}^* = c_i - c_{\alpha}, i \in \overline{\alpha +1, n}$; for such 
$\nu^*, \lambda_1^*, \lambda_2^*$ all kkt is right.

If $\alpha$ is not integer, then we take like first $\lfloor \alpha \rfloor$ elements and difference between $\alpha$ and $\lfloor \alpha \rfloor$. 

$x_i = 1$, $i \in \overline{1, \lfloor \alpha \rfloor - 1}$, $x_{\lfloor \alpha \rfloor } = 1 + \alpha - \lfloor \alpha \rfloor$, $x_{k}$ = 0, $k \in \overline{\lfloor \alpha \rfloor + 1, n}$. $p^* = \sum\limits_{i=1}^{\lfloor \alpha \rfloor - 1}c_i + (1 + \alpha - \lfloor \alpha \rfloor) \cdot c_{\lfloor \alpha \rfloor }$. $\nu^*, \lambda_1^*, \lambda_2^*$ we take from previos task.

If $\mathds{1}^Tx \leq \alpha$, then we can take first $k \leq \alpha$ elements in c such that $c_i < 0$, $i \in \overline{1, k}$, $0 \leq c_{k+1}$. And $p^* = \sum\limits_{i=1}^kc_i$. If all components are positive then we tak $x^* = 0$. $k \geq \alpha$ we take first alpha components if they are negative. And $x_i^* = 1, i \in \overline{1, k}$; $x_i^* = 0, i \in \overline{k+1, n}$

\underline{\textbf{Answer:}} $x^*, \nu^*, \lambda_1^*, \lambda_2^*$ are described above.
\begin{equation*}
    \begin{cases}
    p^* = \sum\limits_{i=1}^\alpha c_i, \alpha \text{ is an integer} \\
    p^* = \sum\limits_{i=1}^{\lfloor \alpha \rfloor}c_i + (\alpha - \lfloor \alpha \rfloor) \cdot c_{\lfloor \alpha \rfloor + 1}, \alpha \text{ is not an integer}  \\
    p^* = \sum\limits_{i=1}^kc_i \text{ ,} \mathds{1}^T x \leq \alpha \text{, the conditions for k are given above (if negatives components exists}, 
    \end{cases}
\end{equation*}

\subsection{Problem №5}
Give an explicit solution of the following QP.

\begin{equation*}
    c^T x  \xrightarrow{} \min\limits_{x \in \mathds{R}^n}
\end{equation*}

\begin{equation*}
   \text{s.t. } x^TAx \leq 1
\end{equation*}

where $A \in \mathds{S}_{++}^n, c \not = 0$. What is the solution if the problem is not convex ($A \not \in \mathds{S}_{++}^n)$. (Hint: consider eigendecomposition of the matrix: $A = Q diag(\lambda)Q^T = \sum\limits_{i=1}^n \lambda_i q_i q_i^T$ and different cases of $\lambda > 0, \lambda = 0, \lambda < 0)$?

\underline{\textbf{Solution:}}
Lagrangian: 
\begin{equation*}
    L(x, \lambda) = c^Tx + \lambda (x^TAx - 1)
\end{equation*}
KKT conditions:
\begin{equation*}
    \begin{cases}
    \nabla_x L = c + (A+A^T)x = c + 2\lambda Ax = 0\\
    \lambda \geq 0 \\
    \lambda (x^TAx - 1) = 0\\
    x^TAx \leq 1
    \end{cases}
\end{equation*}

KKT conditions will be necessary and sufficient, because our problem is convex and Slater's conditions are met. It's right, we can take x such as first component will be equals one divide on ten multiply sum of components of first line of matrix A, and others components of x will be equal zero. From that we get that for such x: $x^TAx = \frac{1}{10} < 1$. And Slater's conditions are met.

\begin{equation*}
    x = -\frac{A^{-1}c}{2\lambda}
\end{equation*}

\begin{equation*}
    \lambda(\frac{c^TA^{-1}c}{4\lambda^2} - 1) = 0
\end{equation*}

\begin{equation*}
    \lambda = \sqrt{\frac{c^TA^{-1}c}{4}}
\end{equation*}
And $x^*$ will be:

\begin{equation*}
    x^* = - \frac{A^{-1}c}{\sqrt{c^TA^{-1}c}}
\end{equation*}

And optimal value will be: $p^* = - \sqrt{c^TA{-1}c}$

If matrix is not positive symmertrical.

If $\lambda_k > 0$, $i \in \overline{1, n}$, then it's the same task.

If $\lambda_k \leq 0$, then we can take $y_k \xrightarrow{} \pm \infty$, and minimum will be $-\infty$.

If $\lambda_k = 0$, if $b_k = 0$ for all k such $\lambda_k = 0$, then it's the same task, but with less $\lambda_l > 0$.

\underline{\textbf{Answer:}} $p^* = - \sqrt{c^TA^{-1}c}$, $x^* = $

\subsection{Problem №6}
Give an explicit solution of the following QP.

\begin{equation*}
    c^T x  \xrightarrow{} \min\limits_{x \in \mathds{R}^n}
\end{equation*}

\begin{equation*}
   \text{s.t. } (x-x_c)^TA(x-x_c) \leq 1
\end{equation*}
where $A \in \mathds{S}_{++}^n, c \not = 0, x_c$

\underline{\textbf{Solution:}}
We will take $y = (x-x_c)$, and write answer, because we can rewrite it as the fifth problem we solved.

\begin{equation*}
y^* = - \frac{A^{-1}c}{\sqrt{c^TA^{-1}c}}\text{, } x^* = x_c - \frac{A^{-1}c}{\sqrt{c^TA^{-1}c}}
\end{equation*}

\begin{equation*}
    p^* = - \sqrt{c^TA^{-1}c} + c^Tx_c
\end{equation*} 

\underline{\textbf{Answer:}} $p^* = - \sqrt{c^TA^{-1}c} + c^Tx_c$, $x^* = x_c - \frac{A^{-1}c}{\sqrt{c^TA^{-1}c}}$

\subsection{Problem №7}
Give an explicit solution of the following QP.

\begin{equation*}
    x^TBx  \xrightarrow{} \min\limits_{x \in \mathds{R}^n}
\end{equation*}

\begin{equation*}
   \text{s.t. } x^TAx \leq 1
\end{equation*}
where $A \in \mathds{S}_{++}^n, B \in \mathds{S}_+^n$

\underline{\textbf{Solution:}}
We know that $B \in \mathds{S}_+^n$, and from that $0 \leq x^TBx$ for any $x \in \mathds{R}^n$, and minimum will be when $x^* = 0$, let's show that $x^*$ is feasible for our conditions $(x^*)^TAx^* = 0 \leq 1$. 
Wohoo!

\underline{\textbf{Answer:}} $p^* = 0$, $x^* = 0$.

\subsection{Problem №8}
Consider the equality constrained least-squares problem 

\begin{equation*}
    ||Ax-b||_2^2  \xrightarrow{} \min\limits_{x \in \mathds{R}^n}
\end{equation*}

\begin{equation*}
   \text{s.t. } Cx = d
\end{equation*}
where $A \in \mathds{R}^{m \dot n}$ with 

\textbf{rank}$A = n$, and $C \in \mathds{R}^{k \dot n}$ with \textbf{rank}$C = k$. Give the KKT conditions, and derive expressions for the primal solution $x^*$ and the dual solution $\lambda^*$.

\underline{\textbf{Solution:}}
If system $Cx = d$ has no solutions, then $p^* = +\infty$

If system $Cx = d$ has solutions, then by Slater's conditions we have an internal point:

\begin{equation*}
    L(x, \nu) = \langle Ax - b, Ax -b \rangle + \nu^T(Cx-d)
\end{equation*}

\begin{equation*}
    \begin{cases}
    \nabla_x L = 2A^T(Ax-b) + \nu^TC = 0 \\ 
    \nabla_{\nu} L = Cx - d= 0
    \end{cases}
\end{equation*}

\begin{equation*}
    \begin{cases}
    x = (A^TA)^{\dag}(A^Tb - \frac{1}{2}\nu^TC) \\
    x = C^{\dag}d
    \end{cases}
\end{equation*}
Then we get: 

\begin{equation*}
(A^TA)^{\dag}(A^Tb - \frac{1}{2}\nu^TC) = C^{\dag}d
\end{equation*}

\begin{equation*}
    A^Tb - \frac{1}{2}\nu^TC = (A^TA)^{\dag})^{\dag} (C^{\dag}d) = (A^TA)(C^{\dag}d)
\end{equation*}
Then we get $\nu^*$
\begin{equation*}
    \nu^* = 2(A^TbC^{\dag} - (A^TA)(C^{\dag}d)C^{\dag})^T
\end{equation*}

\underline{\textbf{Answer:}}

\begin{equation*}
    \begin{cases}
        p^* = +\infty \text{, if system } Cx = d \text{ has no solutions}  \\
        x^* = C^{\dag}d, \nu^* = 2(A^TbC^{\dag} - (A^TA)(C^{\dag}b)C^{\dag})^T \text{, otherwise} \\
    \end{cases}
\end{equation*}

\subsection{Problem №9}
Derive the KKT conditions for the problem 

\begin{equation*}
    trX - \log (detX) \xrightarrow{} \min\limits_{X \in \mathds{S}_{++}^n}
\end{equation*}

\begin{equation*}
   \text{s.t. } Xs = y
\end{equation*}
, where $y \in \mathds{R}^n$ and $s \in \mathds{R}^n$ are given with $y^Ts = 1$. Verify that the optimal solution is given by:
\begin{equation*}
    X^* = I + yy^T - \frac{1}{s^Ts}ss^T
\end{equation*}

\underline{\textbf{Solution:}}
Let's write Lagrangian:
\begin{equation*}
    L(x, \nu) = trX  - \log (detX) + \nu^T(Xs - y)
\end{equation*}
It can be seen that Slaters conditions are met: problem is convex and all functions in conditions are affine.

\begin{equation*}
    \begin{cases}
        \nabla_x L = I - X^{-1} + \nu s^T = 0 \text{ } | \text{ on right }  \cdot y \\ 
        \nabla_{\nu} L = Xs - y = 0
    \end{cases}
\end{equation*}

$X^* = I + yy^T - \frac{1}{s^Ts}ss^T$:

\begin{equation*}
    X^*s - y = Is + yy^Ts - \frac{1}{s^Ts}ss^Ts - y = s + y - s - y = 0
\end{equation*}

Wohoo, it's feasible for KKT! Let's find nu for that point:
\begin{equation*}
\begin{cases}
    y - X^{-1}y + \nu s^T y = 0 \\
    s = X^{-1}y 
    \end{cases}
\end{equation*}

\begin{equation*}
    \begin{cases}
    y - s + \nu s^T y = 0 \text{ } | \text{ on left } \cdot y^T \\
    y^Ty - y^Ts + y^T \nu s^T y = y^Ty - 1 + y^T \nu = 0 | \text{ transpose}
    \end{cases}
\end{equation*}

\begin{equation*}
    \begin{cases}
        \nu y^T  = 1 - y^Ty \text{ } | \text{on right } \cdot s \\
        \nu = s - y^Tys
    \end{cases}
\end{equation*}

We can find $X^{-1}$ from first condition of KKT:
\begin{equation*}
    X^{-1} = I +(s-y^Tys)s^T = I + ss^T - y^Ty ss^T
\end{equation*}
Let's check $X^*X^{-1} = I$
\begin{equation*}
\begin{gathered}
    (I + yy^T - \frac{ss^T}{s^Ts})(I + ss^T - y^Ty ss^T) = 
    I + ss^T - y^Ty ss^T +
    + yy^T  +yy^T ss^T - yy^T y^Ty ss^T  \\
    - \frac{ss^T}{s^Ts} - \frac{ss^Tss^T}{s^Ts} + \frac{ss^T}{ss^T} y^Ty ss^T = I
\end{gathered}
\end{equation*}
Check, that X lie in $\mathds{S}_{++}^n$ is very easy: $I + yy^T - \frac{1}{s^Ts}ss^T$ is positive and symmetric matrix. I want to say that is easy you need to look at matrix and understand, that $I - \frac{1}{s^Ts}ss^T$ is symmetric and 
positive and $yy^T$ is also positive and symmetric, but some will people can not understand that. Okay, let's do it by definition:

\begin{equation*}
    x^T(I + yy^T - \frac{1}{s^Ts}ss^T)x = (x^T + x^Tyy^T - \frac{x^T}{s^Ts}ss^T)x = ...
\end{equation*}

\begin{equation*}
    ... = x^Tx + (y^Tx)^T(y^Tx) - \frac{1}{s^Ts} (s^Tx)^T(s^Tx) = \langle x, x \rangle + \langle y, x \rangle^2 - \frac{\langle s, x\rangle^2}{s^Ts} \geq 0
\end{equation*}
It's right because length of projection on unit vector is less then length of vector in space.

\subsection{Problem №10}
Supporting hyperplane interpretation of KKT conditions. Consider a convex problem with no equality constraints

\begin{equation*}
    f_0(x)  \xrightarrow{} \min\limits_{x \in \mathds{R}^n}
\end{equation*}

\begin{equation*}
   \text{s.t. } f_i(x) \leq 0\text{, } i = \overline{1, m}
\end{equation*}

Assume, that $\exists x^* \in \mathds{R}^n, \mu^* \in \mathds{R}^m$ satisfy the KKT conditions

\begin{equation*}
    \begin{cases}
        \nabla_x L(x^*, \mu^*) = \nabla f_0(x^*) + \sum\limits_{i=1}^m \mu_i^* \nabla f_i(x^*) = 0 \\
        \mu_i^* \geq 0\text{, } i = \overline{1, m} \\
        \mu_i^*f_i(x^*) = 0\text{, } i = \overline{1, n} \\
        f_i(x^*) \leq 0 \text{, } i = \overline{1, n}
    \end{cases}
\end{equation*}
Show that:
\begin{equation*}
    \nabla f_0(x^*)^T(x - x^*) \geq 0 
\end{equation*}
for all feasible x. In other words the KKT conditions imply the simple optimality criterion or $\nabla f_0(x^*)$ defines a supporting hypperplane to the feasible set at $x^*$. 

\underline{\textbf{Solution:}}
We all know that for convex functions it is right that a criterion for feasible $x^*$:
\begin{equation*}
    f_i(x) - f_i(x^*) \geq \nabla f_i(x^*)^T(x-x^*) 
\end{equation*}
Let's multiply it by $\mu_i^*$ and sum over i:
\begin{equation*}
    \sum\limits_{i=1}^{m}f_i(x)\mu_i^* \geq \sum\limits_{i=1}^m f_i(x^*)\mu_i^* + \sum\limits_{i=1}^n \mu_i^* \nabla f_i(x^*)^T(x-x^*)
\end{equation*}

From KKT we know: $\sum\limits_{i=1}^n \mu_i^* \nabla f_i(x^*)^T(x-x^*) = - \nabla f_0(x^*)^T(x-x^*)$,  $f_i(x) \leq 0$ and $f_i^*(x^*)\mu_i^* = 0$.

\begin{equation*}
    0 \leq \sum\limits_{i=1}^{m}f_i(x)\mu_i^* \leq -\nabla f_0(x^*)^T(x-x^*)
\end{equation*}
And we get:
\begin{equation*}
    \nabla f_0(x^*)^T(x-x^*) \geq 0 
\end{equation*}



